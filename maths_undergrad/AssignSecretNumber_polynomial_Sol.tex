\begin{parts}
        \part \textbf{Tommy} \\
            We first observe that $x^3+Ax^2+Bx+C$ has no nonnegative root. \\
            Next, by Rational Root Theorem, we deduce that the only possible roots are $-1$ and $-C$. \\
            Note that if $d$ is repeated integer root, then the third root is also an integer by considering $-A$ is the sum of root. \\
            Therefore we conclude that the polynomial must be of the form $(x+1)^n(x+C)^{3-n}$. \\
            By considering product of roots, we deduce $n=2$. \\
            Hence the polynomial must be of the form $x^3+Ax^2+Bx+C=(x+1)^2(x+C)=x^3+(C+2)x^2+(2C+1)x+C$ if it has repeated integer root. \\
            Therefore, Tommy cannot deduce whether the polynomial has repeated root only when $B$ is in the form of $2p+1$ for some prime $p$. \\
            The above sentence is now a known information to all students. \\
        \part \textbf{David} \\
            If $x^3+Ax^2+Cx+k$ has repeated root $-d$, then by part (a), $3d^2-2dA+C=0$ and $-d^3+Ad^2-Cd+k=0$. \\
            We claim that it can be made to have repeated root $-d=-1$, and it is impossible to have repeated root with $-d\neq\pm1$. \\
            If $d=1$, we need $3-2A+C=0$ and $-1+A-C+k=0$. \\
            That is, $C=2A-3$ and $-1+A-C+k=0$. \\
            If $C$ is odd, then $A$ and $k$ can be chosen to fulfill both conditions, which implies $x^3+Ax^2+Cx+k$ can have repeated root. \\
            Therefore David should not deduce his claim unless $C$ is even, i.e. $C=2$. \\
            The above sentence is now a known information to all students. \\
            Remark: If there is repeated root for other values of $d$, one may easily show that $C$ is non-prime by (1) there are no nonnegative root, (2) relating $C$ and roots by Vieta's formula, contradicting to what we have known. \\
        \part \textbf{Issac} \\
            Since $C=2$, if $x^3+Ax^2+Bx+C$ has integer roots, it must be $-1$ or $-2$ by Rational Root Theorem. \\
            For $x=-1$, we have $-1+A-B+2=0$ and equivalently $A=B-1=2p$. \\
            For $x=-2$, we have $-8+4A-2B+2=0$ and equivalently $A=p+2$. \\
            Since Issac does not know any information about $B$ except it is of the form $2p+1$, we conclude that $A$ is either of the form $2q$ or $q+2$ for some prime $q$. \\
            The above sentence is now a known information to all students. \\
        \part \textbf{Charles} \\
            Using Vieta's formula, we see that sum of root $=-12$, product of roots $=2B$. \\
            Note $A>3$ imply the polynomial has no nonnegative root. \\
            Also, $2B\equiv2 \Mod 4$ implies there is exactly $1$ even root $(\equiv2 \Mod 4)$, and $2$ odd roots. (This is unnecessary, but helps to reduce number of cases) \\
            So the remaining is just brute force computation.\\
            We separate into the following cases: the roots are \\
            $-1,-1,-10$ \\
            $-1,-5,-6$ \\
            $-3,-3,-6$ \\
            $-1,-9,-2$ \\
            $-3,-7,-2$ \\
            $-5,-5,-2$ \\
            For each case, we apply Vieta's formula to compute the corresponding $(A,B)$. \\
            $(23,5)$ \\
            $(43,15)$ \\
            $(47,27)$ \\
            $(31,9)$ \\
            $(43,21)$ \\
            $(47,25)$ \\
            Since $A-2$ or $\dfrac{A}{2}$ is prime, and $B=2p+1$, $(A,B)$ can only be $(43,15)$.\\
            So $(A,B,C)=(43,15,2)$.
        \end{parts}
